\pdfminorversion=7
\documentclass[../main.tex]{subfiles}

\begin{document}
\ifx\chapincluded\undefined
  \begin{refsection}[main-bib]
 \fi

\chapter{Concluding Remarks}
\label{chap:conclusions}
%\section{Conclusion}
Function-as-a-Service (FaaS) is a highly attractive programming model
due to its ease of deployment, zero provisioning overhead for
developers and its ability to compose applications from functions
written in multiple different programming languages. However, the
short-running nature of FaaS functions challenges the assumptions
underpinning the design of conventional
microarchitectures. Additionally, the language-independence,
modularity and deployment flexibility of FaaS functions are enabled by
the use of high-overhead, network-backed RPC interfaces for
inter-function communications. These interfaces add significant
overhead to inter-function communication in FaaS applications which
cause significant performance deterioration. In this thesis, we have
analyzed and optimized the execution of serverless functions from both
the hardware and software perspective. In doing so, we made the
following key contributions to each of our research directions.

\begin{description}


\item[RD-A] We began exploring this direction by answering a key
  question about FaaS functions: How long does a function need to run
  before the warm-up delay of microarchitectural structures is
  amortized? Our analysis shows that only functions that run for a very
  short time (<1 ms) see performance deterioration as a result of
  microarchitectural warm-up latency. We furthermore observe that this
  effect is exacerbated for FaaS functions with a large instruction
  footprint and that the functions in general have a large instruction
  footprint. The latter observation corroborates results from previous
  work and means that FaaS functions are affected by the front-end
  bottleneck in line with general server workloads. To address this we
  introduce BTB-X, an optimized BTB organization that stores branch
  target offsets instead of full branch targets. To improve storage
  density, BTB-X exploits the uneven distribution of branch offset
  sizes in workloads. The resulting design significantly improves BTB
  storage capacity without increasing storage requirements. This, in
  turn, improves the performance of instruction cache prefetchers that
  rely on BTB storage capacity.

\item[RD-B] To address the excessive communication overhead in FaaS
  applications we introduce CoFaaS, a fully automated and
  software-based transformation of FaaS applications. Previous work
  targeting the same problem can effectively reduce the communication
  overhead. However, all previously proposed optimization sacrifices
  at least one of the essential properties that make the FaaS
  programming model attractive, for example by requiring that all
  functions in a FaaS application be written in the same
  language. CoFaaS practically eliminates the inter-function
  communication overhead without sacrificing any of FaaS' essential
  properties. The key insight exploited by CoFaaS is that we can use
  the well-defined RPC interfaces of FaaS functions to make code
  transformations that alleviate the inter-function communication
  overhead. As CoFaaS' mechanism of action consolidates multiple FaaS
  functions onto a single WebAssembly runtime, it also intrinsically
  address the performance deterioration that we observed for very
  short-running functions while working on RD-A.

\end{description}

% \section{Future Work}

% \truls{Write future work}

% Hardware software sections

% BTB-X only evaulates a single level BTB and different startegies may now work the same for all levels
 
% Detail the things mentioned in the CoFaaS papers

%\section{Outlook}

\ifx\chapincluded\undefined
  \printbibliography
  \end{refsection}
 \fi
\end{document}

%%% Local Variables:
%%% mode: latex
%%% TeX-master: t
%%% TeX-command-extra-options: "-shell-escape"
%%% End:
