\pdfminorversion=7
\documentclass[../main.tex]{subfiles}

\begin{document}
\ifx\chapincluded\undefined
  \begin{refsection}[main-bib]
 \fi

\chapter{Concluding Remarks}
\label{chap:conclusions}
%\section{Conclusion}
Function-as-a-Service (FaaS) is a highly attractive programming model
due to its ease of deployment, zero provisioning overhead for
developers and its ability to compose functions written in multiple
different programming languages. However, the short-running nature of
FaaS functions challenges the assumptions underpinning the design of
conventional microarchitectures..Additionally, the
language-independence, modularity and deployment flexibility of FaaS
functions is enabled by the use of high-overhead, network-backed RPC
interfaces for inter-function communications. These interfaces add
significant overhead to inter-function communication in FaaS
applications which cause significant performance deterioration. In
this thesis, we have explored the analysis and optimization of
serverless functions from the hardware and software perspective. In
doing so, we made the following key contributions to each of our
research directions.

\begin{description}

\item[RD-A] We began exploring this direction by answering a key
  question about FaaS functions: How long does a function need to run
  before the warm-up delay of microarchitectural structures is
  amortized? Our analysis show that only functions that run for a very
  short time (<1 ms) see performance deterioration as a result of
  microarchitectural warm-up latency. We furthermore observe that this
  effect is exacerbated for FaaS functions with a large instruction
  footprint and that the functions in general have a large instruction
  footprint. The latter observation corroborates results from previous
  work and means that FaaS functions are affected by the front-end
  bottleneck in line with general server workloads. To address this we
  introduce BTB-X, an optimized BTB organization that stores branch
  target offsets instead of full branch targets. To improve storage
  density, BTB-X exploits \texttt{}he uneven distribution of branch offset
  sizes in workloads. The resulting design significantly improves BTB
  storage capacity without increasing storage requirements. This, in
  turn, improves the performance of instruction cache prefetchers that
  rely on BTB storage capacity.

\item[RD-B] To address the excessive communication overhead of FaaS
  applications we introduce CoFaaS, a fully automated and
  software-based transformation of FaaS applications. Previous work
  targeting the same problem is able to effectively reduce the
  communication overhead. However, all previously proposed
  optimization sacrifices at least on of the essential properties that
  make the FaaS programming model attractive, for example by requiring
  that all functions in a FaaS application are written in the same
  language. CoFaaS practically eliminates the inter-function
  communication overhead without sacrificing any of FaaS' essential
  properties. This uncompromising nature makes CoFaaS unique in the
  solution space.

\end{description}

% \section{Future Work}

% \truls{Write future work}

% Hardware software sections

% BTB-X only evaulates a single level BTB and different startegies may now work the same for all levels
 
% Detail the things mentioned in the CoFaaS papers

%\section{Outlook}

\ifx\chapincluded\undefined
  \printbibliography
  \end{refsection}
 \fi
\end{document}

%%% Local Variables:
%%% mode: latex
%%% TeX-master: t
%%% TeX-command-extra-options: "-shell-escape"
%%% End:
