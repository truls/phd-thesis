\pdfminorversion=7
\documentclass[../main.tex]{subfiles}

\begin{document}
\ifx\chapincluded\undefined
  \begin{refsection}[main-bib]
 \fi

\chapter{Concluding Remarks}
\label{chap:conclusions}
\section{Conclusion}
Function-as-a-Service (FaaS) is a highly attractive programming model
due to its ease of deployment, zero provisioning overhead for
developers and its ability to compose functions written in multiple
different programming languages. However, the short-running nature of
FaaS functions challenges the assumptions that conventional
microarchitecture are designed under which makes them perform less
efficiently. In this thesis, we have explored the analysis and
optimization of serverless functions from the hardware and software
perspective. In doing so, we made the following key contributions to each of our research directions.

\begin{description}

\item[RD-A] We began exploring this direction by answering a key question about FaaS functions: 

\item[RD-B] CoFaaS is a software transformation for FaaS applicaiton that eliminates communication overhea dby 

\end{description}

% \section{Future Work}

% \truls{Write future work}

% Hardware software sections

% BTB-X only evaulates a single level BTB and different startegies may now work the same for all levels
 
% Detail the things mentioned in the CoFaaS papers

%\section{Outlook}

\ifx\chapincluded\undefined
  \printbibliography
  \end{refsection}
 \fi
\end{document}

%%% Local Variables:
%%% mode: latex
%%% TeX-master: t
%%% TeX-command-extra-options: "-shell-escape"
%%% End:
