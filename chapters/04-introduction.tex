\pdfminorversion=7
\documentclass[../main.tex]{subfiles}

\begin{document}
\ifx\chapincluded\undefined
  \begin{refsection}[main-bib]
 \fi


\chapter{Introduction}

\section{Motivation}
Over the past couple of decades, the computing landscape has undergone a radical transformation. Driven by the ubiquity of the internet, computing resources has increasingly shifted from the hands of the user and into Datacenters. This shift has spearheaded an entirely new computing paradigm, Warehouse Scale Computing (WSC)~\cite{barroso18_datac_as_comput}. For most of its existence, a key feature of WSC systems is that they consist of a large aggregation of commodity computing system containing commodity processors~\cite{barroso03_web_searc_planet}. Using commodity hardware components is favored due to their low cost and wide availability. However, scholarly research revealed that commodity processors are fundamentally unfit for the WSC workloads~\cite{ ferdman12_clear_cloud,kanev15_profil}.


\section{Research overview}

At a high level, the research presented in this thesis is motivated by two key characteristics exhibited by serverless applications

Problem 1: They run for a short time which makes it difficult for structures to train

Problem 2: They are highly modular which increases communication overhead

Overall, this thesis makes 3 primary contributions:

\vspace*{0.5cm}

\noindent
\textbf{Primary contribution 1: An in-depth microarchitectural profiling of serverless functions}
Several items of previous work 

\vspace*{0.5cm}

\noindent
\textbf{Primary contribution 2: A storage-efficient BTB organization for servers}

\vspace*{0.5cm}

\noindent
\textbf{primary contribution 3: A software-based optimization methodology for serverless functions.}

\section{Thesis Overview}


\ifx\chapincluded\undefined
  \printbibliography
  \end{refsection}
 \fi

\end{document}

%%% Local Variables:
%%% mode: latex
%%% TeX-master: t
%%% TeX-command-extra-options: "-shell-escape"
%%% End: