\pdfminorversion=7
\documentclass[../main.tex]{subfiles}

\begin{document}
\ifx\chapincluded\undefined
  \begin{refsection}[main-bib]
 \fi

\chapter{Background}
\label{chap:background}

Paper giving background on serverless computing \cite{van2018serverless}

\section{Hardware}

Front-end problem -> what people have done

\subsection{Designing a Microarchitecture for the Datacenter}
\cite{kanev15_profil,ferdman12_clear_cloud}


\subsection{Instruction Delivery in the Datacenter}
\label{sec:instr-delivery}
Be sure to explain how the BTB impacts instruction delivery by aiding FDIP-backed instruction prefetching.

\subsubsection{The Branch Target Buffer}
\label{sec:btb-background}

Two prefetching approaches: FDIP and streaming


FDIP approaches: \cite{reinman99_fetch_direc_instr_prefet, kumar17_boomer,kumar18_blast_throug_front_end_bottl_with_shotg,kumar20_shoot_down_server_front_end_bottl}

Streaming approaches:
\cite{ferdman08_tempor,ferdman11_proac_instr_fetch,kaynak13_shift,kaynak15_confl}

Profiling based:

RDIP-based methods are superior but require highly-efficient prefetchers to work. This motivated the development of BTB-X

Explain diffeernt approaches, prefetching, BTB, cache replacement

Add a description of the processor frontend

\section{Software}

Communication problem -> what people have done

\subsection{Serverless computing: Characteristics and challenges}
\label{sec:serverless}
Serverless is cool, attractive and highly used. What are the characteristics and challenges that makes it unique?

Mention cold start latency

\subsection{Microarchitectural perspective on Serverless computing}


\subsection{Optimization of Serverless Computing}



\ifx\chapincluded\undefined
  \printbibliography
  \end{refsection}
 \fi
\end{document}

%%% Local Variables:
%%% mode: latex
%%% TeX-master: t
%%% TeX-command-extra-options: "-shell-escape"
%%% End: