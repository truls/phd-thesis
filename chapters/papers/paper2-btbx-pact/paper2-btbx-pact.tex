\documentclass[../../../main.tex]{subfiles}

\begin{document}


\chapter{Paper \liningfigures{A3} -- A Specialized BTB Organization for Servers}
\label{chap:pact-paper}

\noindent \textbf{Authors}

\vspace*{0.3cm}

\noindent Truls Asheim, Boris Grot and Rakesh Kumar

\vspace*{0.7cm}

\noindent \textbf{Published in}

\vspace*{0.3cm}

\noindent Proceedings of the International Conference on Parallel Architectures and Compilation Techniques, 2022

\vspace*{0.7cm}

\noindent \textbf{Copyright}

\vspace*{0.3cm}

\noindent Copyright ©2022 The Authors

\newpage

\vspace*{0.1cm}

\begin{center}

\Huge{A Specialized BTB Organization for Servers}

\vspace{0.6cm}

\large{Truls Asheim$^{1}$, Boris Grot$^{2}$, Rakesh Kumar$^{1}$}

\vspace{0.1cm}

\small{1) Norwegian University of Science and Technology, Norway}\\
\small{2) University of Edinburgh, UK}


\end{center}

\vspace{0.2cm}

\begin{center}
  \textbf{Abstract}
  \end{center}
\begin{changemargin}{0.75cm}{0.75cm}
Contemporary server applications feature massive instruction footprints stemming from deeply layered software stacks. These footprints far exceed the capacity of the branch target buffer (BTB) and instruction cache (\liningfigures{L1-I}), resulting in the so-called front-end bottleneck. BTB misses may lead to wrong-path execution, triggering a pipeline flush when misspeculation is detected. Such pipeline flushes not only throw away tens of cycles of work but also expose the fill latency of the pipeline. Similarly, \liningfigures{L1-I} misses cause the core front-end to stall for tens of cycles while the miss is being served from lower-level caches.
\end{changemargin}

\vspace{1cm}

%-------------------------------------------------------------------------
\begin{refsection}[btbx-pact]
%\begin{refsection}[chapters/papers/paper2-btbx-pact/refs]
  \pdfminorversion=7
\documentclass[10pt,b5paper,twoside,openright,onecolumn]{thesis}

\def\fronttitle{Analyzing and Optimizing\\ Serverless Function Execution}
\def\doctitle{Analyzing and Optimizing Serverless Function Execution}
\def\docauthor{Truls Asheim}

% Library includes


\usepackage[utf8]{inputenc}
  \IfFileExists{MinionPro.sty}{
    \IfFileExists{MyriadPro.sty}{
      \usepackage[textosf,mathlf]{MinionPro}
      \usepackage[textosf]{MyriadPro}
      \usepackage[T1]{fontenc}
      \usepackage{amsthm}
      \usepackage{bm}
      \usepackage{MnSymbol}
      \def\taFontsIncluded{true}
    }{}}{}
  \ifx\taFontsIncluded\undefined
    \usepackage[T1]{fontenc}
    \usepackage[p,osf]{newtxtext}
    \usepackage[vvarbb]{newtxmath}
    \useosf
  \fi

 \usepackage[scaled=0.75]{beramono}

 \usepackage{float}
\usepackage{graphicx}
% \usepackage{tabularx}
\usepackage{rotating}
\usepackage{subcaption}
\usepackage{appendix}
\usepackage{fancyhdr}
\usepackage{styles}
\usepackage{scrextend}

\usepackage{placeins}
\usepackage{minted}
%\setminted{fontsize=\footnotesize}
\setminted{fontsize=\small,baselinestretch=1}
%\usepackage[draft]{todonotes}
\usepackage[final]{todonotes}
\presetkeys%
    {todonotes}%
    {inline,backgroundcolor=white}{}
\usepackage{multirow}
\usepackage{paralist}
\usepackage{csquotes}
\usepackage{dblfloatfix}
\usepackage{fontaxes}
\usepackage[english]{babel}
\RequirePackage[backend=biber,sorting=nyt, style=numeric-comp,eprint=false, url=false,natbib,backref=true]{biblatex}
\usepackage[pdfauthor={\docauthor},
            pdftitle={\doctitle}]{hyperref}
            \def\UrlBreaks{\do\/\do-}
\usepackage[nameinlink,sort]{cleveref}



\renewcommand\chaptermark[1]{\markboth{\MakeUppercase{\liningfigures{\thechapter}.~#1}}{}}

\newcommand{\Description}[1]{}

\usepackage{caption}
\captionsetup{font=small,labelfont=bf}

% Custom commands
\def\changemargin#1#2{\list{}{\rightmargin#2\leftmargin#1}\item[]}
\let\endchangemargin=\endlist

\newcommand\magnus[1]{\noindent{\todo{\color{blue} {\bf \fbox{magnus}} {\it#1}}}}
\newcommand\truls[1]{\noindent{\todo{\color{purple} {\bf \fbox{truls}} {#1}}}}
\newcommand\rakesh[1]{\noindent{\todo{\color{red} {\bf \fbox{rakesh}} {\it#1}}}}

\defbibheading{refs}{\section{References}}

\addbibresource[label=main-bib]{./bibliography.bib}
%\addbibresource{./bibliography.bib}
\addbibresource[label=btbx-cal]{chapters/papers/paper1-btbx-cal/refs.bib}
\addbibresource[label=btbx-pact]{chapters/papers/paper2-btbx-pact/refs.bib}
\addbibresource[label=faas-profiling]{chapters/papers/paper3-faas-profiling/refs.bib}
\addbibresource[label=btbx-hpca]{chapters/papers/paper4-btbx-hpca/refs.bib}
\addbibresource[label=cofaas]{chapters/papers/paper5-cofaas/bibliography.bib}

\usepackage{subfiles}

\begin{document}

\subfile{chapters/00-coverpage}
\subfile{chapters/01-abstract}
\subfile{chapters/02-preface}
\subfile{chapters/03-acknowledgements}


\tableofcontents
%\listoftodos
\newpage

\mainmatter


\part{Research Overview}
\label{part:one}
%\truls{Use lining figures for all section numbers}
\def\chapincluded{true}
\begin{refsection}[main-bib]
\subfile{chapters/04-introduction}
\subfile{chapters/05-background}
%\subfile{chapters/06-Methodology}
\subfile{chapters/07-contributions}
%\subfile{chapters/08-discussion}
\subfile{chapters/09-conclusion}

\printbibliography

\end{refsection}



% \bibliographystyle{unsrt}
% \bibliography{references.bib}



\part{Publications}
\label{part:two}
\subfile{chapters/papers/paper3-faas-profiling/paper3-faas-profiling}
\subfile{chapters/papers/paper1-btbx-cal/paper1-btbx-cal}
\subfile{chapters/papers/paper2-btbx-pact/paper2-btbx-pact}
\subfile{chapters/papers/paper4-btbx-hpca/paper4-btbx-hpca}
\subfile{chapters/papers/paper5-cofaas/paper5-cofaas}


% \appendix
% \input{appendices/a-appendix.tex}

\end{document}


%%% Local Variables:
%%% mode: latex
%%% TeX-master: t
%%% TeX-command-extra-options: "-shell-escape"
%%% End:
  \printbibliography[heading=refs]
\end{refsection}
\FloatBarrier
\end{document}

%%% Local Variables:
%%% mode: latex
%%% TeX-master: t
%%% TeX-command-extra-options: "-shell-escape"
%%% End:
