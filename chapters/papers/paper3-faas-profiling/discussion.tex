\section{Discussion}
\label{wosc:sec:discussion}


The results of our study shows that functions with very short execution times (< 1 ms) benefit significantly from being executed on a processor with a warm microarchitectural state. However, as noted in the introduction, such short functions are quite uncommon in real-world applications. Additionally, even if the microarchitectural efficiency of these functions are improved using targeted optimizations the running time of the actual function may still only constitute a small fraction of the total round trip time of a function invocation. Meanwhile, larger functions, whose execution time contribute significantly to the total request round trip time, will not see any benefit from targeted microarchitectural optimizations.

These observations motivates research into high-level approaches for improving the execution time of the shortest functions. For example, functions that are executed in sequence as part of an application graph could be dynamically compiled together into a single component. Such an approach would preserve the function compositionality and modularity that are essential to the serverless application model while, at the same time, eliminate the overhead arising from the execution of a large number of short functions on both the microarchitectural and system level.

However, it is important to note that even if the execution overhead of serverless application graphs can be reduced by dynamically compiling functions into a single component, a large number of serverless applications consist only of a single function \cite{serverless_state}. Furthermore, many of these requests are interactive, that is, the user that made the request expects an immediate response. The functions responding to such requests have an inherently short running time and are therefore likely to still see significant benefit from targeted microarchitectural optimizations.


%%% Local Variables:
%%% mode: latex
%%% TeX-master: "main"
%%% End:
