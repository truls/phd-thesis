\section{Conclusion}
\label{sec:conclusion}

This paper aimed to identify properties of serverless functions that predicts if a function is likely to benefit from warm microarchitectural state. To do this, we evaluated a suite of both real-world and synthetic functions to identify their per-invocation execution times and their instruction working set sizes. Subsequently we interleaved the function executions with a process that thrashes the microarchitectural state of the previously invoked function. By comparing the performance of the back-to-back and interleaved function executions across several metrics we identified key properties that makes a function likely to benefit from being executed from a warm microarchitectural state. We found that only the functions with a very short execution time (< 1 ms) and large instruction working sets are negatively affected by being interleaved with the thrasher. However, functions with longer execution times (> 50 ms) were not adversely affected by the microarchitectural state thrashing.



%%% Local Variables:
%%% mode: latex
%%% TeX-master: "main"
%%% End:
